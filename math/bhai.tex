\subsection{Combinatorics}
\vspace{2.5mm}
\begin{itemize}
    \item \begin{math}
        \sum\limits_{k =0}^n\binom{n-k}{k} = Fib_{n+1}
    \end{math}

    \item \begin{math}
        \binom{n}{k} + \binom{n}{k+1} = \binom{n+1}{k+1}
    \end{math}

    \item \begin{math}
        k \binom{n}{k}  = n\binom{n-1}{k-1}
    \end{math}
    \item Number of binary sequences of length n such that no two 0's are adjacent = $Fib_{n+1}$
    \item Number of non-negative solution of $x_1 + x_2 + x_3 + ... + x_k = n$ is $\binom{n+k-1}{n}$
\end{itemize}
\subsubsection{Catalan Number}
\vspace{2.5mm}
\begin{itemize}
    \item $C_n = \frac{1}{n+1}\binom{2n}{n} = \binom{2n}{n}-\binom{2n}{n+1} = \frac{(2n)!}{(n+1)!n!}$
    \item $C_0 = 1, C_1 = 1, C_n = \sum_{k=0}^{n-1}C_kC_{n-1-k}$
    \item $1, 1, 2, 5, 14, 42, 132, 429, 1430, 4862, 16796, 58786$
    \item Number of correct bracket sequences consisting of n opening brackets.
    \item Number of ways to completely parenthesize n+1 factors.
    \item The number of triangulations of a convex polygon with +2 sides (i.e. the number of partitions of polygon into disjoint triangles by using the diagonals).
    \item The number of ways to connect the 2n points on a circle to form n disjoint i.e. non-intersecting chords.
    \item The number of monotonic lattice paths from point (0,0) to point (n,n) in a square lattice of size $n\times n$, which do not pass above the main diagonal
    \item Number of permutation of length n that can be stack sorted.
    \item The number of non-crossing partitions of a set of n elements.
    \item The number of rooted full binary tree with n+1 leaves.
    \item The number of Dyck words of length 2n. A string consisting of n X's and n Y's such that no string prefix has more Y's than X's.
    \item Number of permutation of length n with no three-term increasing subsequence.
    \item Number of ways to tile a stairstep shape of height n with n rectangle.
    \item $C^k_n = \frac{k+1}{n+1}\binom{2n-k}{n-k}$ denote the number of bracket sequences of size 2n with the first k elements being (.
    \item $N(n,k) = \frac{1}{n}\binom{n}{k}\binom{n}{k-1}$
    \item The number of expressions containing n pairs of correct parentheses, which contain k distinct nestings.
    $N(4,2) = 6$ \\
    $()((())), (())(()), (()(())), ((()())), ((())()), ((()))()$
    \item The number of paths from (0,0) to (2n, 0) with steps only northeast and southeast, not staying below the x-axis with k peaks. And sum of all number of peaks is Catalan number.
\end{itemize}
\subsubsection{Stirling Number of the First Kind}
\vspace{2.5mm}
\begin{itemize}
    \item Count permutation according to their number of cycles.
    \item $S(n,k)$ count the number of permutation of n elements with k disjoint cycles.
    \item $S(n,k) = (n-1) \times S(n-1,k) + S(n-1, k-1), S(0,0) = 1, S(n,0) = S(0,n) = 0$
    \item $S(n,1) = (n-1)!$
    \item $S(n,n-1) = \binom{n}{2}$
    \item $\sum_{k=0}^n S(n,k) = n!$
\end{itemize}

\subsubsection{Stirling Numbers of the Second Kind}
\vspace{2.5mm}
\begin{itemize}
    \item Number of ways to partition a set of n objects into k non-empty subsets.
    \item $S(n,k) = k*S(n-1,k)+S(n-1,k-1), S(0,0) = 1, S(n,0) = S(0,n) = 0$
    \item $S(n,2) = 2^{n-1}-1$
    \item $S(n,k) = \frac{1}{k!}\sum\limits_{j=0}^{k}(-1)^{k-j}\binom{k}{j}j^n$
   
    \item $S(n,k)*k!$ = number of ways to color n nodes using colors from 1 to k such that each color is used at least once.
\end{itemize}
\subsubsection{Bell Number}
\vspace{2.5mm}
\begin{itemize}
    \item Counts the number of partitions of a set.
    \item $B_{n+1} = \sum_{k=0}^n \binom{n}{k}*B_k$
    \item $B_n = \sum_{k=0}^n S(n,k)$, where S is Stirling number of second kind.
    \item The number of multiplicative partitions of a square free number with i prime factors is the i-th Bell number.
    \item $B(p^m+n) \equiv mB(n)+B(n+1) (\mod p)$
    \item If a deck is shuffled by removing and reinserting the top card n times, there are $n^n$ possible shuffles. The number of shuffles that return the deck to its original order is $B_n$, so the probability of returning to the original order is $B_n/n^n$.
\end{itemize}
\subsubsection{Lucas Theorem}
\vspace{2.5mm}
\begin{itemize}
    \item If p is prime then $\binom{p^a}{k}\equiv 0\mod p$
    \item For non-negative integers m and n and a prime p:\\
    $\binom{m}{n} = \prod\limits_{i=0}^k \binom{m_i}{n_i}(\mod p)$ where \\
    $m = m_kp^k+m_{k-1}p^{k-1}+...+m_1p+m_0$
    $n = n_kp^k+n_{k-1}p^{k-1}+...+n_1p+n_0$
    are the base p expansion.
\end{itemize}
\subsubsection{Derangement}
\vspace{2.5mm}
\begin{itemize}
    \item A permutation such that no element appears in its original position.
    \item $d(n) = (n-1)*(d(n-1)+d(n-2)),d(0) = 1, d(1) = 0$
    \item $d(n) =nd(n-1)+(-1)^n = \lfloor{\frac{n!}{e}}\rfloor, n\geq1$
\end{itemize}
\subsubsection{Burnside Lemma}
\vspace{2.5mm}
Given a group G of symmetries and a set X, the number of elements of X up to symmetry equals
$$\frac{1}{|G|} \sum\limits_{g \in G} |X^g|$$ where $X^g$ are the elements fixed by $g (g.x = x)$
If f(n) counts "configurations" of some sort of length n, we can ignore rotational symmetry using $G = \mathbb{Z}_n$ to get
$$g(n) = \frac{1}{n} \sum\limits_{k=0}^{n-1}f(gcd(n,k))= \frac{1}{n}\sum\limits_{k|n}f(k)\phi(n/k)$$

\subsubsection{Eulerian Number}
\vspace{2.5mm}
\begin{itemize}
    \item $E(n,k)$ is the number of permutations of the numbers 1 to n in which exactly k elements are greater than the previous element.
    \item $E(n,k) = (n-k)E(n-1, k-1)+(k+1)E(n-1,k), E(n,0) = E(n,n-1) = 1$
    \item $E(n,k) = \sum\limits_{j=0}^k (-1)^j \binom{n+1}{j}(k+1-j)^n$
    \item $E(n,k) = E(n, n-1-k)$
    \item $E(0,k) = [k=0]$
    \item $E(n,1) = 2^n - n - 1$
\end{itemize}

\subsection{Number Theory}
\vspace{2.5mm}
\subsubsection{Mobius Function and Inversion}
\vspace{2.5mm}
\begin{itemize}
    \item define $\mu(n)$ as the sum of the primitive nth roots of unity depending on the factorization of n into prime factors:
    \[
\mu(x) =
\begin{cases}
0 & \text{n is not square free} \\
1 & \text{n has even number of prime factors}\\
-1 & \text{n has odd number of prime factors}
\end{cases}
\]
\item Mobius Inversion:
$$g(n) = \sum_{d|n}f(d)\leftrightarrow f(n) = \sum_{d|n}\mu(d)g(n/d)$$
\item $\sum\limits_{d|n}\mu(d) = [n=1]$
\item $\phi(n) = \sum\limits_{d|n} \mu(d).\frac{n}{d} = n\sum\limits_{d|n}\frac{\mu(d)}{d} = \sum\limits_{d|n}d.\mu(\frac{n}{d})$
\item $a|b \rightarrow \phi(a)|\phi(b)$
\item $\phi(mn) = \phi(m).\phi(n).\frac{d}{\phi(d)}$ where $d = gcd(m,n)$
\item $\phi(n^m) = n^{m-1}\phi(n)$
\item $\sum\limits_{i=1}^n[gcd(i,n)=k] = \phi(\frac{n}{k})$
\item $\sum\limits_{i=1}^n gcd(i,n) = \sum\limits_{d|n}d.\phi(\frac{n}{d})$

\item $\sum\limits_{i=1}^n \frac{1}{gcd(i,n)} = \sum\limits_{d|n}\frac{1}{d}.\phi(\frac{n}{d}) = \frac{1}{n}\sum\limits_{d|n}d.\phi(d)$

\item $\sum\limits_{i=1}^n \frac{i}{gcd(i,n)} = \frac{n}{2}.\sum\limits_{d|n}\frac{1}{d}.\phi(\frac{n}{d}) = \frac{n}{2}.\frac{1}{n}\sum\limits_{d|n}d.\phi(d)$

\item $\sum\limits_{i=1}^n \frac{n}{gcd(i,n)} = 2.\sum\limits_{i=1}^n \frac{i}{gcd(i,n)} - 1$
\end{itemize}
\subsubsection{GCD and LCM}
\vspace{2.5mm}
\begin{itemize}
    \item gcd(a,b) = gcd(b, a mod b)
    \item If $a | b.c$, and gcd(a,b) = d, then $(a/d)|  c$.
    \item GCD is a multiplicative function.
    \item gcd(a, lcm(b,c)) = lcm(gcd(a,b), gcd(a,c))
    \item $gcd(n^a - 1, n^b - 1) = n^{gcd(a,b)} - 1$
\end{itemize}

\subsubsection{Gauss Circle Theorem}
\vspace{2.5mm}
\begin{itemize}
    \item Determine the number of lattice points in a circle centered at the origin with radius r.
    \item number of pairs (m,n) such that $m^2+n^2\leq r^2$
    \item $N(r) = 1+4\sum\limits_{i=0}^{\infty}(\lfloor\frac{r^2}{4i+1}\rfloor - \lfloor\frac{r^2}{4i+3}\rfloor)$
\end{itemize}
\subsubsection{Pick's Theorem}
\vspace{2.5mm}
According to Pick’s Theorem We can calculate the area of any polygon by just counting the number of Interior and Boundary lattice points of that polygon. If number of interior points are I and number of boundary lattice points are B then Area (A) of polygon will be:$$Area = I + B/2 - 1$$where I is the number of points in the interior shape, B stands for the number of points on the boundary of the shape.
\subsubsection{Formula Cheatsheet}
\vspace{2.5mm}
\begin{itemize}
    \item $\sum\limits_{i=1}^n = \frac{1}{m+1}[(n+1)^{m+1} - 1 - \sum\limits_{i=1}^n((i+1)^{m+1}-i^{m+1}-(m+1)i^m)]$

    \item $\sum\limits_{i=0}^nc^i = \frac{c^{n+1}-1}{c-1}, c\neq 1$

    \item $\sum\limits_{i=0}^\infty c^i = \frac{1}{1-c}, \sum\limits_{i=1}^\infty c^i = \frac{c}{1-c}, |c|< 1$

    \item $H_n = \sum\limits_{i=1}^n \frac{1}{n}, \sum\limits_{i=1}^n iH_i = \frac{n(n+1)}{2}H_n - \frac{n(n-1)}{4}$

    \item $\sum\limits_{k=0}^n\binom{r+k}{k} =\binom{r+n+1}{n}$
   
\end{itemize}
